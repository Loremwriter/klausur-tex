\documentclass[a4paper,11pt, headsepline,headinclude]{scrreprt}
\usepackage{ifthen}
\usepackage[ngerman]{babel}
\usepackage{amsmath}
\usepackage[utf8]{inputenc}
\usepackage[T1]{fontenc}
\usepackage{lastpage}
\usepackage{tabularx}
\usepackage{scrpage2}
\usepackage[]{graphicx}
\usepackage{paralist}
\usepackage[babel,german=quotes]{csquotes}
\usepackage{geometry}
\usepackage[math]{iwona} %hübsche Schrift
\usepackage[decimalsymbol=comma]{siunitx} % Einheiten so: \SI{3.4}{kg} -> 3,4 kg
\usepackage{tikz}



\geometry{a4paper, portrait, inner=2.5cm, outer=2cm, top=2.5cm, bottom=2cm}

\newcounter{AufgabenCounter}
\setcounter{AufgabenCounter}{1}

\newcounter{Gruppenwahl}

% Jeweils eine der folgenden Zeilen auskommentieren:
\setcounter{Gruppenwahl}{1}  % Gruppe A
%\setcounter{Gruppenwahl}{2}  % Gruppe B

\newcommand{\AB}[2]%
{%
\ifthenelse{\equal{\theGruppenwahl}{1}}{#1}{}%
\ifthenelse{\equal{\theGruppenwahl}{2}}{#2}{}%
}
% Verwendung Gruppe A rechnet 1+2 und Gruppe B rechnet exp(x)
% \AB{$1+2$}{$\exp(x)$} 

\pagestyle{scrheadings}
\newcommand{\Thema}{Ganzrationale Funktionen}
\newcommand{\Datum}{11. Januar 2014}
\newcommand{\Kurs}{Mathematik GK\,12}
\newcommand{\Klausurnummer}{2. Klausur}
\newcommand{\Gruppe}{Gruppe \Alph{Gruppenwahl}}
\newcommand{\Seite}{Seite \thepage/\pageref{LastPage}}


% Nummer und Punkte selbst angeben
\newcommand{\AufgabeNrPkt}[2]{\vspace{5mm} \textbf{#1) \hfill (#2 Punkte)} }

% Nummer selbst angeben, aber keine Anzeige von Punkten 
\newcommand{\AufgabeNr}[1]{\vspace{5mm} \textbf{#1)} }

% Nummer automatisch und keine Anzeige von Punkten 
\newcommand{\Aufgabe}{%
\AufgabeNr{Aufgabe \theAufgabenCounter}%
\stepcounter{AufgabenCounter}}

% Nummer automatisch, Anzeige von Punkten 
\newcommand{\AufgabePkt}[1]{%
\AufgabeNrPkt{Aufgabe \theAufgabenCounter}{#1}%
\stepcounter{AufgabenCounter}}


\setlength{\parindent}{0pt} % Absatzeinrückung von Links

\begin{document}
\lehead{Name:}
\lohead{Name:\\ \Kurs}
\rehead{\Klausurnummer, \Datum \\ \Gruppe, \Seite}
\rohead{\Klausurnummer, \Datum \\ \Gruppe, \Seite}

\cfoot{} 
% leere Fußzeile, damit keine Seitenzahl in der Fußzeile erscheint. 
% Wir haben ja eine in der Kopfzeile

\cohead{\large{\Thema}}


\AufgabePkt{6, 7}
\begin{enumerate}[a)]
\item Gegeben sind die drei Punkte $A(1|1)$, $B(-4|5)$ und $C(3|6\frac{3}{4})$. Durch die drei 
Punkte soll eine Parabel verlaufen.

Schreiben Sie den Ansatz auf, mit dem sich die Parabelgleichung bestimmen lässt.
\item Bestimmen Sie die Lösungsmenge des angegebenen lineare Gleichungssystems mit Hilfe des 
Gaußschen Eliminationsverfahrens:

$\begin{vmatrix} 3x-y+4z=12 \\ x-y+z=4 \\ 2x-8y+10z=30 \end{vmatrix}$
\end{enumerate}

\Aufgabe

Gegeben sind die folgenden ganzrationalen Funktionen.

Zeigen Sie jeweils, dass es sich tatsächlich um ganzrationale Funktionen handelt und 

geben Sie jeweils die Symmetrie und den Globalverlauf an. Begründen Sie Ihre Angaben.
\begin{enumerate}[a)]
 \item $f(x)= \frac{3x^5-18x^3-21x}{3}$
 \item $g(x)= (x+7)^3$
 \item $h(x)= \frac{-5x^8+0,5x^4-8x^2}{x^2}$
\end{enumerate}

\AufgabeNrPkt{Bonusaufgabe}{8, 3}
\begin{enumerate}[a)]
 \item Gegeben sind die folgenden ganzrationalen Funktionen. Bestimmen Sie jeweils die Nullstellen.
  \begin{enumerate}[1.]
   \item $f(x)= \frac{5x^3-5x^2-10x}{5}$
   \item $g(x)= x(x+5)(x-4)(x+0,5)$
  \end{enumerate}
 \item Geben Sie an, welche der drei Aussagen richtig ist und begründen Sie Ihre Entscheidung.

Geben Sie auch Gegenbeispiele gegen die falschen Aussagen an.
 \begin{enumerate}[1.]
  \item Sarah:  \enquote{Eine ganzrationale Funktion vom Grad n hat höchstens n Nullstellen.}
  \item Julian: \enquote{Eine ganzrationale Funktion vom Grad n hat genau n Nullstellen.}
 \end{enumerate}

\end{enumerate}



{\bigskip {\large \textsl{Viel Erfolg!}}}

\end{document}
