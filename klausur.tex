\documentclass[a4paper,11pt,headsepline,headinclude]{scrreprt}

\usepackage[ngerman]{babel}
\usepackage{amsmath}
\usepackage[utf8]{inputenc}
\usepackage[T1]{fontenc}
\usepackage{lastpage}
\usepackage{tabularx}
\usepackage{scrpage2}
\usepackage[]{graphicx}
\usepackage{paralist}
\usepackage[babel,german=quotes]{csquotes}
\usepackage{geometry}
\newcommand{\Aufg}[3]{% Nr. Punkte 
            \vspace{5mm} \textbf{Aufgabe #1  \hfill (#2 Punkte)} }

\geometry{a4paper, portrait, inner=2.5cm, outer=2cm, top=2.5cm, bottom=2cm}

\pagestyle{scrheadings}
\newcommand{\Thema}{Ganzrationale Funktionen}
\newcommand{\Datum}{11. Januar 2014}
\newcommand{\Kurs}{Mathematik GK\,12}
\newcommand{\Klausurnummer}{2. Klausur, Seite \thepage/\pageref{LastPage}}

\setlength{\parindent}{0pt} % Absatzeinrückung von Links

\begin{document}
\lehead{Name:}
\lohead{Name:\\ \Kurs}
\rehead{\Klausurnummer \\ \Datum }
\rohead{\Klausurnummer \\ \Datum }

\cohead{\Thema}
%\cefoot{Seite \thepage/\pageref{LastPage}}
%\cofoot{Seite \thepage/\pageref{LastPage}}

\Aufg{1)}{6, 7}\\
\begin{enumerate}[a)]
\item Gegeben sind die drei Punkte $A(1|1)$, $B(-4|5)$ und $C(3|6\frac{3}{4})$. Durch die drei 
Punkte soll eine Parabel verlaufen.

Schreiben Sie den Ansatz auf, mit dem sich die Parabelgleichung bestimmen lässt und erläuteren Sie ihn.
\item Bestimmen Sie die Lösungsmenge des angegebenen lineare Gleichungssystems mit Hilfe des 
Gaußschen Eliminationsverfahrens:

$\begin{vmatrix} 3x-y+4z=12 \\ x-y+z=4 \\ 2x-8y+10z=30 \end{vmatrix}$
\end{enumerate}

\Aufg{2)}{6, 6, 6}\\
%\Aufg{Eigenschaften ganzrationaler Funktionen bestimmen}{7, 7, 7 Punkte}

Gegeben sind die folgenden ganzrationalen Funktionen.

Zeigen Sie jeweils, dass es sich tatsächlich um ganzrationale Funktionen handelt und 

geben Sie jeweils die Symmetrie und den Globalverlauf an. Begründen Sie Ihre Angaben.
\begin{enumerate}[a)]
 \item $f(x)= \frac{3x^5-18x^3-21x}{3}$
 \item $g(x)= (x+7)^3$
 \item $h(x)= \frac{-5x^8+0,5x^4-8x^2}{x^2}$
\end{enumerate}

\Aufg{3)}{8, 3}\\
\begin{enumerate}[a)]
 \item Gegeben sind die folgenden ganzrationalen Funktionen. Bestimmen Sie jeweils die Nullstellen.
  \begin{enumerate}[1.]
   \item $f(x)= \frac{5x^3-5x^2-10x}{5}$
   \item $g(x)= x(x+5)(x-4)(x+0,5)$
  \end{enumerate}
 \item Geben Sie an, welche der drei Aussagen richtig ist und begründen Sie Ihre Entscheidung.

Geben Sie auch Gegenbeispiele gegen die falschen Aussagen an.
 \begin{enumerate}[1.]
  \item Sarah:  \enquote{Eine ganzrationale Funktion vom Grad n hat höchstens n Nullstellen.}
  \item Julian: \enquote{Eine ganzrationale Funktion vom Grad n hat genau n Nullstellen.}
 \end{enumerate}

\end{enumerate}



{\bigskip {\Huge Viel Erfolg!}}


\end{document}
